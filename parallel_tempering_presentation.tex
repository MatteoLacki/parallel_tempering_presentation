\documentclass[xetex]{beamer}
%\documentclass[draft, xetex]{beamer}

\usepackage{mystyle}
\usepackage{my_commands}

\mode<presentation>{
	%\usetheme{Berlin}
	\usetheme{CambridgeUS}
	\setbeamercovered{transparent}
}

\setmainfont[Ligatures=TeX]{Linux Libertine O}


	% Title
\title[Parrellel Tempering]{Parrellel Tempering }
\subtitle{theory and applications}

%\beamerdefaultoverlayspecification{<+->}
\date{24 April 2013}
\author[Łącki]{Mateusz Łącki}
\institute[UW]{Uniwersytet Warszawski}
\titlegraphic{\includegraphics[scale=.4, keepaspectratio]{./picts/eagle.jpg}}

\usefonttheme[onlylarge]{structuresmallcapsserif}
%\usefonttheme[onlysmall]{structurebold}


	

\begin{document}
\fontspec[Numbers={OldStyle}]{Linux Libertine O}


\begin{frame}
\titlepage
\end{frame}

\begin{frame}
	\frametitle{Plan prezentacji}
	\tableofcontents[pausesections]
\end{frame}



\section[Bayes]{Przypomnienie filozofii rozumowania Bayes'owskiego}

\subsection{Teoria}

\begin{frame}
	\frametitle{Model Statystyczny}
	
\begin{itemize}
	\item $X $ charakterystyki pewnego zjawiska.
	\item $$X: \underbrace{(\Omega, \mathcal{F})}_{\text{stany natury}}\rightarrow \underbrace{(\mathcal{X}, \mathcal{B})}_{\text{realizacje zjawiska}} $$
	\item Mierzalność: $ X^{(-1)} [ \mathcal{B}] = \{ X^{(-1)} [A] : A \in \mathcal{B} \} \subset \mathcal{F}$
\end{itemize} 	


\begin{columns}
\begin{column}[t]{5cm}
	\begin{itemize}
		\item \emph{Nie} $( \Omega, \mathcal{F}, \mathbf{P} )$ \dots
		\item \dots ale $( \mathcal{X}, \mathcal{B}, \{\mathbf{P_\theta} : \theta \in \Theta\} )$.
		\item
		\item $\{\mathbf{P_\theta} : \theta \in \Theta\} \subset \mathcal{M} $
		\item $\mathcal{M}$ - miary probablistyczne
	\end{itemize}
\end{column}
\begin{column}[t]{5cm}
	\begin{center}
		HEllo	
	\end{center}		
\end{column}
\end{columns}
		
\end{frame}



\begin{frame}
	\frametitle{Under construction}
\end{frame}


\section[Bibliografia]{Bibliografia}

\begin{frame}

	\frametitle{Hello Kitchen}
	
	
	\begin{thebibliography}{10}

		\beamertemplatebookbibitems
		
			\bibitem{promo}
	  			Błażej Miasojedow, Eric Moulines, Matti Vihola,
	 			\emph{Adaptive Parallel Tempering Algorithm},
	  			Arxive.

		\beamertemplatearticlebibitems
		
			\bibitem{geyer}  
				Charles J. Geyer,
				\emph{Markov Chain Monte Carlo Lecture Notes}.  

	\end{thebibliography}

\end{frame}

\end{document}
  
  
  %%%%%%%%%%%%%%%%%%%%%%%%%%%%%%%%%%%%%%%%%%%%%%%%%%%%%%%%%%%%%%%%%%%%%%%%%%%

		\beamertemplatearticlebibitems
	
		\bibitem{Glimm}
			X. Li , J. Glimm , X. Jiao, Ch. Peyser , Y. Zhao (2010)
			\newblock{ \it Study of crystal growth and solute precipitation through front tracking method} 
			\newblock{Acta Mathematica Scientia 30B(2): 377-390}



			\item Funkcja kowariancji $$\Gamma(\psi, \phi) = \mathbb{E} X_{\psi} X_{\phi} = \HI{\psi}{\phi}$$
			\item Pozornie ograniczyliśmy się do 				
		\end{itemize}
		
		
		\begin{block}{Twierdzenie}
			Funkcja kowariancji jest pół-dodatniookreślona $\sum{i,j} \Gamma()$
		\end{block}
		
		
		
\begin{frame}
	\frametitle{Znajdźmy kernel}
	
		\begin{itemize}
			\item $\mathcal{J} := \mathbb{H}$ (rezygnujemy z literki $\mathcal{I}$)
		\end{itemize}
		
		\begin{block}{Twierdzenie}
			Dowolna przestrzeń Hilberta jest izomorficzna z $\mathrm{l}^2(A)$ dla pewnego $\mathcal{A}$:
			$$ \mathrm{l}^2(A) = \Big\{ \mathbf{z} \in \mathbb{C}^\mathcal{A}: \int_{A} z_a \mathrm{d} \#(a) = \sum_{a \in \mathcal{A}} \mathbf{z}_a < \infty \Big\}$$
		\end{block}
				
		
		\begin{itemize}
			\item Weźmy $\mathbb{H} = \mathbb{L}^2(\mathcal{I}, \mathcal{B}, \mu)$.	
			\item Npx. $\mathcal{I} = \mathbb{R}^D \ni \xxx_i$		
			\item Wtedy $\Gamma(\xxx, \xxx') = \mathbb{E}X(\phi(\xxx))X(\phi(\xxx')) = \HI{\phi(\xxx)}{\phi(\xxx')} = \int_{\mathcal{I}} \phi()$	
		\end{itemize}	

\end{frame}
