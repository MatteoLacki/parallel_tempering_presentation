\documentclass[xetex]{beamer}
%\documentclass[draft, xetex]{beamer}


		% Iclude packages and commands used text-wide.	
%%%%%%%%%%%%%%%%%%%%%%%%%%%%%%%%%%%%%%%%%%%%%%%%%%%%%%%%%%%%%%%%%%%%%%%%%%%%%%%%%%%%%%%%%%%%%%%%%%%%%%%%%%%

\usepackage{mystyle}
\usepackage{my_commands}


		% Presentation settings.
%%%%%%%%%%%%%%%%%%%%%%%%%%%%%%%%%%%%%%%%%%%%%%%%%%%%%%%%%%%%%%%%%%%%%%%%%%%%%%%%%%%%%%%%%%%%%%%%%%%%%%%%%%%

	\mode<presentation>{
		%\usetheme{Berlin}
		\usetheme{CambridgeUS}
		\setbeamercovered{transparent}
	}




		% Title
	\title[Parrellel Tempering]{Parrellel Tempering }
	\subtitle{theory and applications}

	%\beamerdefaultoverlayspecification{<+->}
	\date{24 April 2013}
	\author[Łącki]{Mateusz Łącki}
	\institute[UW]{Uniwersytet Warszawski}
	\titlegraphic{\includegraphics[scale=.4, keepaspectratio]{./picts/eagle.jpg}}

	\usefonttheme[onlylarge]{structuresmallcapsserif}
	%\usefonttheme[onlysmall]{structurebold}


		% The document
%%%%%%%%%%%%%%%%%%%%%%%%%%%%%%%%%%%%%%%%%%%%%%%%%%%%%%%%%%%%%%%%%%%%%%%%%%%%%%%%%%%%%%%%%%%%%%%%%%%%%%%%%%%	

\begin{document}
\fontspec[Numbers={OldStyle}]{Linux Libertine O}


	\begin{frame}
		\titlepage
	\end{frame}

	\begin{frame}
		\frametitle{Today's Agenda}
		%\tableofcontents[pausesections]
		\tableofcontents
	\end{frame}


	\subsection{Teoria}

\begin{frame}
		\frametitle{Model Statystyczny}
	
	\begin{itemize}
		\item $X $ charakterystyki pewnego zjawiska.
		\item $$X: \underbrace{(\Omega, \mathcal{F})}_{\text{stany natury}}\rightarrow \underbrace{(\mathcal{X}, \mathcal{B})}_{\text{realizacje zjawiska}} $$
		\item Mierzalność: $ X^{(-1)} [ \mathcal{B}] = \{ X^{(-1)} [A] : A \in \mathcal{B} \} \subset \mathcal{F}$
	\end{itemize} 	


	\begin{columns}
	\begin{column}[t]{5cm}
		\begin{itemize}
			\item \emph{Nie} $( \Omega, \mathcal{F}, \mathbf{P} )$ \dots
			\item \dots ale $( \mathcal{X}, \mathcal{B}, \{\mathbf{P_\theta} : \theta \in \Theta\} )$.
			\item
			\item $\{\mathbf{P_\theta} : \theta \in \Theta\} \subset \mathcal{M} $
			\item $\mathcal{M}$ - miary probablistyczne
		\end{itemize}
	\end{column}
	\begin{column}[t]{5cm}
		\begin{center}
			HEllo	
		\end{center}		
	\end{column}
	\end{columns}
		
\end{frame}


	



\begin{frame}
	\frametitle{Under construction}
\end{frame}


\section[Bibliografia]{Bibliografia}

\begin{frame}

	\frametitle{Hello Kitchen}
	
	
	\begin{thebibliography}{10}

		\beamertemplatebookbibitems
		
			\bibitem{promo}
	  			Błażej Miasojedow, Eric Moulines, Matti Vihola,
	 			\emph{Adaptive Parallel Tempering Algorithm},
	  			Arxive.

		\beamertemplatearticlebibitems
		
			\bibitem{geyer}  
				Charles J. Geyer,
				\emph{Markov Chain Monte Carlo Lecture Notes}.  

	\end{thebibliography}

\end{frame}

\end{document}
  
  
%%%%%%%%%%%%%%%%%%%%%%%%%%%%%%%%%%%%%%%%%%%%%%%%%%%%%%%%%%%%%%%%%%%%%%%%%%%

